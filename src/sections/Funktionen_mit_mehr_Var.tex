\section{Funktionen mit mehreren Variablen}

\textbf{Differenzierbarkeit:}\\
Es gilt: \\
Jede nicht abschnittsweise definierte Funktion, welche aus den Grundrechenarten (e, ln, sin, cos, ...) zusammengesetzt ist, ist differenzierbar.

\textbf{Höhenlinen:}
\begin{align*}
    f(x,y) = c \quad \text{mit} \quad c \in \mathbb{R}
\end{align*}

\textbf{Gradient:}
\begin{align*}
    grad_f(x,y) &= \nabla f(x,y) = \left( \begin{array}{c}
        f_x(x,y)\\
        f_y(x,y)
    \end{array} \right) \\
    grad_f(x_0,y_0) &= \nabla f(x_0,y_0) = \left( \begin{array}{c}
        f_x(x_0,y_0)\\
        f_y(x_0,y_0)
    \end{array} \right) \\
\end{align*}
\textbf{Jakobimatrix:}
\begin{align*}
    J_f(x,y) &= \left( \begin{array}{cc}
        f_x(x,y) & f_y(x,y)
    \end{array} \right) \\
    J_f(x_0,y_0) &= \left( \begin{array}{cc}
        f_x(x_0,y_0) & f_y(x_0,y_0)
    \end{array} \right)
\end{align*}

\textbf{Tangeltialebene:}
\begin{align*}
    T_f=f(x_0,y_0) + f_x(x_0,y_0)(x-x_0) + f_y(x_0,y_0)(y-y_0)
\end{align*}

\textbf{Richtungsableitung:}
\begin{align*}
    \frac{\partial z}{\partial \vec{v}} (x_0,y_0) = \nabla f(x_0,y_0) \circ \vec{v_0} = \nabla f(x_0,y_0) \circ \frac{1}{|\vec{v}|} \vec{v}
\end{align*}

\textbf{Differentiationsregeln:}
\begin{align*}
    &\lambda \cdot f + \mu \cdot g \rightarrow J_{\lambda f + \mu g}(x,y) = \lambda \cdot J_f(x,y) + \mu \cdot J_g(x,y) \quad &\text{Linearität}\\
    &f \cdot g \rightarrow J_{f \cdot g}(x,y) = f(x,y) \cdot J_g(x,y) + g(x,y) \cdot J_f(x,y) \quad &\text{Produktregel}\\
    &f \circ g \rightarrow J_{f \circ g}(a) = J_f(g(a)) \cdot J_g(a) \quad &\text{Kettenregel}\\
\end{align*}

\textbf{Extremwerte:}
\begin{align*}
    &H_f = \left( \begin{array}{cc}
        f_{xx}(x,y) & f_{xy}(x,y)\\
        f_{xy}(x,y) & f_{yy}(x,y)
    \end{array} \right) \\
    &det(H_f) = f_{xx}(x,y) \cdot f_{yy}(x,y) - (f_{xy}(x,y))^2 \\
    \text{Notwendige Bedingung: } &f_x(x_0,y_0) = 0 \quad \text{und} \quad f_y(x_0,y_0) = 0 \quad \Rightarrow \quad x_0 \text{ und } y_0\\
    &det(H_f(x_0,y_0)) < 0 \quad \text{(Sattelpunkt)} \\
    &det(H_f(x_0,y_0)) > 0 \quad \text{und} \quad f_{xx} > 0 \quad \text{(Minimum)} \\
    &det(H_f(x_0,y_0)) > 0 \quad \text{und} \quad f_{xx} < 0 \quad \text{(Maximum)} \\
\end{align*}
Wichtig: Auch Rand des Definitionsbereich beachten!
\\
\\
\\
\textbf{Normgebiete:}
\begin{figure}[H]
    \includegraphics[width=\textwidth]{Normgebiete.png}
\end{figure}
\begin{align*}
    &\text{1: } \quad \int_{x=a}^{b} \int_{y=\phi_1}^{\phi_2} f(x,y) \, dy \, dx \\
    &\text{2: } \quad \int_{y=c}^{d} \int_{x=\Psi_1}^{\Psi_2} f(x,y) \, dx \, dy \\
    &\text{3: } \quad \int_{\varphi = \alpha}^{\beta} \int_{r = g_1}^{g_2} f(r \cdot \cos(\varphi), r \cdot \sin(\varphi)) r \, dr \, d\varphi \\
\end{align*}

\textbf{Flächeninhalt:}
\begin{align*}
    A &= \int_{x=a}^{b} \int_{y=\phi_1}^{\phi_2} 1 \, dy \, dx = \int_{y=c}^{d} \int_{x=\Psi_1}^{\Psi_2} 1 \, dx \, dy = \int_{\varphi = \alpha}^{\beta} \int_{r = g_1}^{g_2} r \, dr \, d\varphi
\end{align*}

\textbf{Flächenschwerpunkt:}
\begin{align*}
    &f(x,y) \text{ sei eine Dichte-Funktion, so ist der Schwerpunkt } (x_S, y_S) \\
    &x_S = \frac{1}{A} \int \int_{A}^{} x \cdot f(x,y) \, dA \qquad y_S = \frac{1}{A} \int \int_{A}^{} y\cdot f(x,y) \, dA \\
\end{align*}